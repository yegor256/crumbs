% \iffalse meta-comment
% (The MIT License)
%
% Copyright (c) 2021-2025 Yegor Bugayenko
%
% Permission is hereby granted, free of charge, to any person obtaining a copy
% of this software and associated documentation files (the 'Software'), to deal
% in the Software without restriction, including without limitation the rights
% to use, copy, modify, merge, publish, distribute, sublicense, and/or sell
% copies of the Software, and to permit persons to whom the Software is
% furnished to do so, subject to the following conditions:
%
% The above copyright notice and this permission notice shall be included in all
% copies or substantial portions of the Software.
%
% THE SOFTWARE IS PROVIDED 'AS IS', WITHOUT WARRANTY OF ANY KIND, EXPRESS OR
% IMPLIED, INCLUDING BUT NOT LIMITED TO THE WARRANTIES OF MERCHANTABILITY,
% FITNESS FOR A PARTICULAR PURPOSE AND NONINFRINGEMENT. IN NO EVENT SHALL THE
% AUTHORS OR COPYRIGHT HOLDERS BE LIABLE FOR ANY CLAIM, DAMAGES OR OTHER
% LIABILITY, WHETHER IN AN ACTION OF CONTRACT, TORT OR OTHERWISE, ARISING FROM,
% OUT OF OR IN CONNECTION WITH THE SOFTWARE OR THE USE OR OTHER DEALINGS IN THE
% SOFTWARE.
% \fi

% \CheckSum{0}
%
% \CharacterTable
%  {Upper-case    \A\B\C\D\E\F\G\H\I\J\K\L\M\N\O\P\Q\R\S\T\U\V\W\X\Y\Z
%   Lower-case    \a\b\c\d\e\f\g\h\i\j\k\l\m\n\o\p\q\r\s\t\u\v\w\x\y\z
%   Digits        \0\1\2\3\4\5\6\7\8\9
%   Exclamation   \!     Double quote  \"     Hash (number) \#
%   Dollar        \$     Percent       \%     Ampersand     \&
%   Acute accent  \'     Left paren    \(     Right paren   \)
%   Asterisk      \*     Plus          \+     Comma         \,
%   Minus         \-     Point         \.     Solidus       \/
%   Colon         \:     Semicolon     \;     Less than     \<
%   Equals        \=     Greater than  \>     Question mark \?
%   Commercial at \@     Left bracket  \[     Backslash     \\
%   Right bracket \]     Circumflex    \^     Underscore    \_
%   Grave accent  \`     Left brace    \{     Vertical bar  \|
%   Right brace   \}     Tilde         \~}

% \GetFileInfo{crumbs.dtx}
% \DoNotIndex{\endgroup,\begingroup,\let,\else,\fi,\newcommand,\newenvironment}

% \iffalse
%<*driver>
\ProvidesFile{crumbs.dtx}
%</driver>
%<package>\NeedsTeXFormat{LaTeX2e}
%<package>\ProvidesPackage{crumbs}
%<*package>
[0000/00/00 0.0.0 Inputable Shell Executions]
%</package>
%<*driver>
\documentclass{ltxdoc}
\usepackage[tt=false, type1=true]{libertine}
\usepackage{crumbs}
\usepackage{href-ul}
\PageIndex
\EnableCrossrefs
\CodelineIndex
\RecordChanges
\begin{document}
	\DocInput{crumbs.dtx}
	\PrintChanges
	\PrintIndex
\end{document}
%</driver>
% \fi

% \title{|crumbs|: \LaTeX{} Package \\ for Navigation Crumbs\thanks{The sources are in GitHub at \href{https://github.com/yegor256/crumbs}{yegor256/crumbs}}}
% \author{Yegor Bugayenko \\ \texttt{yegor256@gmail.com}}
% \date{\filedate, \fileversion}
%
% \maketitle
%
% \section{Introduction}
%
% This package helps you add navigation crumbs to your document,
% which is most useful for presentations and slide decks:
%\iffalse
%<*verb>
%\fi
\begin{verbatim}
\documentclass{article}
\usepackage{crumbs}
\usepackage{fancyhdr}
  \fancyhf{}
  \pagestyle{fancy}
  \fancyhead[L]{\crumbs / \subcrumbs}
\begin{document}
\section{Introduction}
Some text here.
\section{Related Works}
Some other text here.
\end{document}
\end{verbatim}
%\iffalse
%</verb>
%\fi

% \DescribeMacro{\crumb}
% You may redefine |\crumb{}| and |\subcrumb{}|
% commands, if you want your crumbs to look nicer, for example:
%\iffalse
%<*verb>
%\fi
\begin{verbatim}
\renewcommand\crumb[2]{{
  \ifnum\value{section}=\value{crumbi}
    \color{orange}
  \fi
  #1
}}
\end{verbatim}
%\iffalse
%</verb>
%\fi
% The command will be called by |\crumbs| with two arguments. The first one
% will be the short name of the crumb, the second one will be the long one.
% The counter |crumbi| will be equal to the position of the crumb
% being rendered, starting from zero.

% \DescribeMacro{\subcrumb}
% The same for |\subcrumb{}|, but the comparison should be
% done not between |section| and |crumbi|, but between
% |subsection| and |subcrumbi| counters.

% \DescribeMacro{\section}
% \DescribeMacro{\subsection}
% The first parameter of both |\crumb{}| and |\subcrumb{}|
% is the optional parameter of |\section{}| and |\subsection{}|
% calls respectively. The second one is the mandatory parameter of them, which
% is the full text of the section and sub-section.
% When you use |crumbs| package, you can use |\section| and |\subsection| with
% an optional argument, which will be used in the crumbs, e.g.:
%\iffalse
%<*verb>
%\fi
\begin{verbatim}
\section[Details]{More Details About The Method}
\subsection[SLR]{Systematic Literature Review}
\end{verbatim}
%\iffalse
%</verb>
%\fi

% \section{Package Options}

% \DescribeMacro{dry}
% You may turn on the ``dry'' mode, which will prohibit the package to
% renew |\section{}| and |\subsection{}| commands.
% You will still be able to use the crumbs mechanism through
% |\crumbection{}| and |\subcrumbection{}| commands, for example:
%\iffalse
%<*verb>
%\fi
\begin{verbatim}
\documentclass{article}
\usepackage[dry]{crumbs}
\begin{document}
\crumbection[Intro]{Introduction} % Add it to \crumbs
\section{Introduction} % Print it to the page
Hello, world!
\end{document}
\end{verbatim}
%\iffalse
%</verb>
%\fi

% \DescribeMacro{increment}
% You can also instruct it to increment
% |\section{}| and |\subsection{}|
% on each call to |\crumbection{}| and |\subcrumbection{}|
% respectively. Just use the |increment| package option.

% \section{How It Works}

% On the first run, a new file |jobname.crumbs| is created,
% where all |\section| and |\subsection| commands
% are being logged.

% On the second run, the file is loaded and two commands are being
% created: |\crumbs| and |\subcrumbs|. The first
% one is a collection of |\crumb{}| calls, while the second
% one is a collection of |\subcrumb{}| ones.

% \StopEventually{}

% \section{Implementation}

% \changes{v0.2.0}{2021/07/10}{Initial version}

% \iffalse
\RequirePackage{xkeyval}
\RequirePackage{etoolbox}
\makeatletter\newif\ifcrumbs@dry\makeatother
\makeatletter\DeclareOption{dry}{\crumbs@drytrue}\makeatother
\makeatletter\newif\ifcrumbs@increment\makeatother
\makeatletter\DeclareOption{increment}{\crumbs@incrementtrue}\makeatother
\ProcessOptions\relax
% \fi

% \begin{macro}{\crumbs}
% It is implemented with the help of |catchfile| package. First,
% we define special commands for special characters inside the
% |jobname.crumbs| files, as suggested \href{https://tex.stackexchange.com/a/12414/1449}{here}:
%    \begin{macrocode}
\makeatletter\newcommand\crumbs@back{\@backslashchar}\makeatother
%    \end{macrocode}
% \end{macro}

% \begin{macro}{\appendtofile}
% Then, we define a new command |\appendtofile|:
%    \begin{macrocode}
\RequirePackage{catchfile}
  \newwrite\appendwrite
  \ExplSyntaxOn\cs_new:Npn\appendtofile#1#2{%
    \begingroup
    \IfFileExists{#1}%
      {\CatchFileDef{\filecontent}{#1}{\endlinechar=`^^J\catcode\endlinechar=12\relax}}% keep existing end-of-lines
      {\let\filecontent\empty}%
    \immediate\openout\appendwrite=#1\relax
    \immediate\write\appendwrite{\unexpanded\expandafter{\filecontent} \text_purify:n{#2}}%
    \immediate\closeout\appendwrite
    \endgroup
  }\ExplSyntaxOff
%    \end{macrocode}
% \end{macro}

% \begin{macro}{\subcrumbs}
% The command |\subcrumbs| prints all sub-crumbs for the current subsection:
%    \begin{macrocode}
\newcommand\subcrumbs{%
  \ifnum\value{section}=0\else%
    \expandafter\ifcsname crumbs\romannumeral\the\value{section}\endcsname%
      \csname crumbs\romannumeral\the\value{section}\endcsname{}%
    \fi%
  \fi%
}
%    \end{macrocode}
% \end{macro}

% \begin{macro}{\crumbection}
% The command |\crumbection| registers a new crumb:
%    \begin{macrocode}
\makeatletter\newcommand\crumbection[2][]{%
  \def\temp{#1}%
  \ifx\crumbs\empty%
    \appendtofile{\jobname.crumbs}{%
      \crumbs@back ifx\crumbs@back crumbs\crumbs@back empty%
        \crumbs@back gappto\crumbs@back crumbs\{\crumbs@back setcounter\{crumbi\}\{0\}\}%
      \crumbs@back fi%
      \crumbs@back gappto\crumbs@back crumbs\{\crumbs@back stepcounter\{crumbi\}\crumbs@back crumb{\ifx\temp\empty \{#2\}\else \{#1\}\fi}{\{#2\}}\}%
    }%
  \fi%
  \ifcrumbs@increment\stepcounter{section}\fi%
}\makeatother
%    \end{macrocode}
% \end{macro}

% \begin{macro}{\subcrumbection}
% The command |\subcrumbection| registers a new subcrumb:
%    \begin{macrocode}
\makeatletter\newcommand\subcrumbection[2][]{%
  \def\temp{#1}%
  \ifx\crumbs\empty%
    \appendtofile{\jobname.crumbs}{%
      \crumbs@back ifcsname crumbs\romannumeral\the\value{section}\crumbs@back endcsname\crumbs@back else
        \crumbs@back expandafter\crumbs@back newcommand\crumbs@back csname crumbs\romannumeral\the\value{section}\crumbs@back endcsname\{%
          \crumbs@back setcounter\{subcrumbi\}\{0\}%
        \}%
      \crumbs@back fi%
      \crumbs@back expandafter\crumbs@back gappto\crumbs@back csname crumbs\romannumeral\the\value{section}\crumbs@back endcsname\{%
        \crumbs@back stepcounter\{subcrumbi\}%
        \crumbs@back subcrumb{\ifx\temp\empty \{#2\}\else \{#1\}\fi}{\{#2\}}\}%
    }%
  \fi%
  \ifcrumbs@increment\stepcounter{subsection}\fi%
}\makeatother
%    \end{macrocode}
% \end{macro}

% \begin{macro}{\section}
% We also redefined |\section| and |\subsection| commands:
%    \begin{macrocode}
\makeatletter\ifcrumbs@dry%
  \message{Since "dry" mode is ON we don't renew "section" and "subsection" commands}%
\else%
  \NewCommandCopy\crumbs@oldsection\section
  \renewcommand\section[2][]{%
    \crumbection[#1]{#2}%
    \crumbs@oldsection{#2}%
  }
  \NewCommandCopy\crumbs@oldsubsection\subsection
  \renewcommand\subsection[2][]{%
    \subcrumbection[#1]{#2}%
    \crumbs@oldsubsection{#2}%
  }
\fi\makeatletter
%    \end{macrocode}
% \end{macro}

% \begin{macro}{\crumb}
% We define default implementations of |\crumb| command:
%    \begin{macrocode}
\ifcsname crumb\endcsname%
  \message{The command "crumb" is already defined}%
\else%
  \newcommand\crumb[2]{#1}%
\fi%
%    \end{macrocode}
% \end{macro}

% \begin{macro}{\subcrumb}
% We define default implementations of |\subcrumb| command:
%    \begin{macrocode}
\ifcsname subcrumb\endcsname%
  \message{The command "subcrumb" is already defined}%
\else%
  \newcommand\subcrumb[2]{#1}%
\fi%
%    \end{macrocode}
% \end{macro}

% \begin{macro}{crumbi}
% We also define a few counters:
%    \begin{macrocode}
\newcommand*\crumbs{}
\newcounter{crumbi}
\newcounter{subcrumbi}
%    \end{macrocode}
% \end{macro}

% Finally, we load the file into the document:
%    \begin{macrocode}
\AtBeginDocument{\IfFileExists{\jobname.crumbs}{\input{\jobname.crumbs}}{}}
%    \end{macrocode}

% \Finale

%\clearpage
%
%\PrintChanges
%\clearpage
%\PrintIndex
